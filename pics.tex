% Vertical centring in math mode
\makeatletter% allows \m@th for example
\newcommand\vctr[1]{\vcenter{\hbox{$\m@th{#1}$}}}
\makeatother

% narrow column. argument e.g. A \\ B \\ C
\newcommand\column[1]{\begin{array}{@{}c@{}}#1\end{array}}


\newcommand\skewlifting[1]{\widetilde{#1}}

% Vertex metavariables
\newcommand\vertex{v}
\newcommand\vertexp{{\vertex\mkern-1.3mu'}}
\newcommand\vertexa{w}
\newcommand\vertexaa{u}
\newcommand\vertexap{w\primed}

% Fibration symbol
\newcommand\fib{\phi}




%%%% Connectives

% Logical connectives

\renewcommand{\implies}{\Rightarrow}% overrides ams package
\newcommand\tightwedgeveeshrink{\mkern-2mu}
\newcommand\tightvee{\mathbin{\tightwedgeveeshrink\vee\tightwedgeveeshrink}}
\newcommand\tightwedge{\mathbin{\tightwedgeveeshrink\wedge\tightwedgeveeshrink}}
\newcommand\shortimplies{\mathop{\mkern-0mu\implies\mkern-1mu}}
\newcommand\tightimplies{\shortimplies}
\newcommand\tighteq{\mathop{\mkern-.5mu=\mkern-.5mu}}
\newcommand\tightneq{\mathop{\mkern-.5mu\neq\mkern-.5mu}}
\newcommand\tightin{\mathbin{\mkern-5mu{}\in{}\mkern-5mu}}
\newcommand\tightnotin{\mathbin{\mkern-5mu{}\notin{}\mkern-3mu}}
\newcommand\tightsubseteq{\mathbin{\mkern-.5mu\subseteq\mkern-.5mu}}
\newcommand\tightsupseteq{\mathbin{\mkern-.5mu\supseteq\mkern-.5mu}}
\newcommand\tightsetminus{\mathbin{\mkern-2mu\setminus\mkern-1mu}}

% Graph/cotree connectives

\newcommand\tightplus{\mathbin+}
\newcommand\tighttimes{\mathbin\times}
\newcommand\graphunion{\tightplus}
\newcommand\graphjoin{\tighttimes}



% FORMULAS

% to be able to control the spacing of 'px', 'py' etc:
\newcommand\p{p}\newcommand\pp{\dual p}\newcommand\ppp{\dual\pp}
\newcommand\q{q}\newcommand\qq{\dual q}
\newcommand\px{\p x}
\newcommand\py{\p y}
\newcommand\pxy{\p x y}
\newcommand\ppx{\pp x}
\newcommand\ppy{\pp y}
\newcommand\pa{\p a}
\newcommand\pb{\p b}
\newcommand\ppa{\pp a}
\newcommand\ppb{\pp b}
\newcommand\pz{\p z}
\newcommand\ppz{\pp z}
\newcommand\axpx{\forall x\mkern2.2mu\px}
\newcommand\aypy{\forall y \mkern3mu \py}
\newcommand\qab{\q a b}
\newcommand\qba{\q b a}
\newcommand\qqab{\qq a b}
\newcommand\qqba{\qq b a}
\newcommand\qx{\q x}
\newcommand\qy{\q y}
\newcommand\qqx{\qq x}
\newcommand\qqy{\qq y}
\newcommand\qxy{\q x y}
\newcommand\fx{f\mkern-1.5mu x}
\newcommand\fy{f\mkern-1.5mu y}
\newcommand\fz{f\mkern-1.5mu z}
\newcommand\pfx{\p\mkern-1mu \fx}
\newcommand\qfz{\q \fz}
\newcommand\qqfz{\qq \fz}
\newcommand\pfy{\p\mkern-2mu \fy}
\newcommand\ffy{f\mkern-4mu \fy}
\newcommand\pffy{\p\mkern-2mu \ffy}
\newcommand\ppffy{\pp\mkern-2mu \ffy}
\newcommand\allx{\forall x}
\newcommand\ally{\forall y}
\newcommand\existsx{\exists x}
\newcommand\existsy{\exists y}
\newcommand\rx{rx}
\newcommand\rz{rz}




% Hacks to control height/depth of atom/var labels on vertices

\newcommand{\nohang}[1]{\raisebox{0ex}[\height][0ex]{$#1$}}
\newcommand\likex[1]{\raisebox{0ex}[1ex][0ex]{$#1$}}
\newlength\fheight\settoheight\fheight{$f$}
\newlength\pbarheight\settoheight\pbarheight{$\pp$}
\newcommand\likef[1]{\raisebox{0ex}[\the\fheight][0ex]{$#1$}}
\newcommand\likepbar[1]{\raisebox{0ex}[\the\pbarheight][0ex]{$#1$}}



%%%%%% VERTICES %%%%%%


%%%  Vertex styles

% Vertex colours 

\colorlet{dblue}{black!15!blue}% outline of blue vx
\colorlet{lblue}{blue!8}% fill of blue vx
\colorlet{dred}{black!55!red}% outline of red vx
\colorlet{lred}{red!40}% fill of red vx
\colorlet{dgreen}{black!65!green}% outline of green vx
\colorlet{lgreen}{green!30}% fill of green vx

\newcommand\roundvxwidth{3.6pt}

\tikzstyle{all vertices} = [inner sep=0pt, outer sep=0pt, draw]
\tikzstyle{round vx}     = [all vertices, circle, minimum width=\roundvxwidth, semithick]
\newcommand\squarewidth{4.8pt}
\tikzstyle{square vx}    = [all vertices, regular polygon, regular polygon sides = 4, minimum width=\squarewidth, minimum height=\squarewidth, semithick]
\tikzstyle{diamond vx}   = [square vx, shape border rotate = 45]

\tikzstyle{black vx}     = [round vx, draw=black, fill=black]
\tikzstyle{blue vx}      = [round vx, draw=dblue, fill=lblue]
\tikzstyle{red vx}       = [square vx, draw=dred, fill=lred]
\tikzstyle{green vx}     = [diamond vx, draw=dgreen, fill=lgreen]

\newcommand\vertextoitslabelgap{1.4pt}
\newcommand\labelvertexgap{5pt}
\newcommand\vertexlabelgap{\labelvertexgap}
\newcommand\labellabelgap{5pt}% account for \inlineoutersep
\newcommand\edgepairvxgap{1.8ex}
\tikzset{% for inline labelled graphs
  graphlabel/.style={
	text height=1.4ex,
	text depth =.5ex,
	inner sep=0pt,
	outer sep=0pt,
        anchor=base
  }
}


\newcommand\vx[2]{% #1 placement e.g. "3,4"  #2 name
\node[black vx] ({#2}) at ({#1}) {};}

\newcommand\bluevx[2]{% #1 placement eg "3,4"  #2 name
\node[blue vx] ({#2}) at ({#1}) {};}

\newcommand\redvx[2]{% #1 placement eg "3,4"  #2 name
\node[red vx] ({#2}) at ({#1}) {};}

\newcommand\greenvx[2]{% #1 placement eg "3,4"  #2 name
\node[green vx] ({#2}) at ({#1}) {};}

\newcommand\collvxsep[6]{% #1 placement e.g. "3,4"  #2 name  #3 contents  #4 angle  #5 extra label sep  #6 vertex style 
\node[#6, label={[outer sep=0pt,inner sep=0,label distance={#5}]{#4}:${#3}$}] ({#2}) at ({#1}) {};}

\newcommand\lvxsep[5]{% #1 placement e.g. "3,4"  #2 name  #3 contents  #4 angle  #5 extra label sep
\collvxsep{#1}{#2}{#3}{#4}{#5}{black vx}}
%\node[black vx, label={[label distance={#5}]{#4}:${#3}$}] ({#2}) at ({#1}) {};}
\newcommand\redlvxsep[5]{% #1 placement e.g. "3,4"  #2 name  #3 contents  #4 angle  #5 extra label sep
\collvxsep{#1}{#2}{#3}{#4}{#5}{red vx}}
\newcommand\bluelvxsep[5]{% #1 placement e.g. "3,4"  #2 name  #3 contents  #4 angle  #5 extra label sep
\collvxsep{#1}{#2}{#3}{#4}{#5}{blue vx}}

\newcommand\lvxgap{3pt}
\newcommand\lvx[4]{% #1 placement e.g. "3,4"  #2 name  #3 contents  #4 angle
\lvxsep{#1}{#2}{#3}{#4}{\lvxgap}}
\newcommand\redlvx[4]{% #1 placement e.g. "3,4"  #2 name  #3 contents  #4 angle
\redlvxsep{#1}{#2}{#3}{#4}{\lvxgap}}
\newcommand\bluelvx[4]{% #1 placement e.g. "3,4"  #2 name  #3 contents  #4 angle
\bluelvxsep{#1}{#2}{#3}{#4}{\lvxgap}}


\newcommand\lvxugap{\lvxgap}% specialized to y with its underhang
\newcommand\lvxl[3]{%  left-labelled vertex: #1 coords eg "3,4"  #2 name  #3 contents
\lvxsep{#1}{#2}{\likex{#3}}{180}{\lvxgap}}
\newcommand\lvxr[3]{% right-labelled vertex: #1 coords eg "3,4"  #2 name  #3 contents
\lvxsep{#1}{#2}{\likex{#3}}{0}{\lvxgap}}
\newcommand\lvxusep[4]{% #1 placement e.g. "3,4"  #2 name  #3 contents  #4 sep
\lvxsep{#1}{#2}{#3}{90}{#4}}
\newcommand\lvxu[3]{% #1 placement e.g. "3,4"  #2 name  #3 contents
\lvxusep{#1}{#2}{#3}{\lvxugap}}
\newcommand\lvxd[3]{% #1 placement e.g. "3,4"  #2 name  #3 contents
\lvxsep{#1}{#2}{#3}{-90}{\lvxgap}}


%%% Drawing vertices inline in text, coloured/labelled

\newcommand\namedvx[2][black]{% optional style, name
\node (#2) [#1 vx] {};}

\newcommand\dummystart{\node (start) [graphlabel, inner sep=0, outer sep=0] {};}
\newcommand\vxsep[5][black]{% #1 optional colour, #2 node name, #3 right/left of, #4 separation, #5 right/left
  \node (#2) [#1 vx, #5=#4 of #3] {};}
\newcommand\vxrightsep[4][black]{% #1 optional colour, #2 node name, #3 right of, #4 separation
  \vxsep[#1]{#2}{#3}{#4}{right}}
\newcommand\vxleftsep[4][black]{% #1 optional colour, #2 node name, #3 left of, #4 separation
  \vxsep[#1]{#2}{#3}{#4}{left}}
\newcommand\vxright[3][black]{% #1 optional colour, #2 node name, #3 node name right of
  \vxrightsep[#1]{#2}{#3}{\vertextoitslabelgap}}
\newcommand\vxleft[3][black]{% #1 optional colour, #2 node name, #3 node name left of
  \vxleftsep[#1]{#2}{#3}{\vertextoitslabelgap}}
\newcommand\labelright[4][\vertextoitslabelgap]{% #1 optional sep, #2 label name, #3 label contents, #4 node name right of
  \node (#2) [graphlabel, right=#1 of #4] {$#3$};}
\newcommand\labelleft[4][\vertextoitslabelgap]{% #1 optional sep, #2 label name, #3 label contents, #4 node name left of
  \node (#2) [graphlabel, left=#1 of #4] {$#3$};}
% \newcommand\labelrightgap[3]{% #1 label name, #2 label contents, #3 node name right of
%   \node (#2) [graphlabel, right=\labellabelgap of #3] {$#2$};}

% \newcommand\vxsolo[2][black]{% #1 optional colour, #2 name
% \dummystart\vxright[#1]{#2}{start}}
\newcommand\labelsolo[2]{% #1 name, #2 contents
\dummystart\labelright[-1pt]{#1}{#2}{start}}% \node (#1) [graphlabel] {$#2$};   todo: why do we need this -1pt to make it flush?
\newcommand\labvxright[4][black]{% #1 optional colour, #2 vertex name, #3 label name, #4 vertex type (optional)
  % \labelsolo {#3} {#4}
  % \vxleft[#1] {#2} {#3}
  \dummystart % use the dummy start so that the vertex aligns correctly (center of vertex horizontally aligned with center of 'x')
  \vxrightsep[#1] {#2} {start} {0}
  \labelright {#3} {#4} {#2}
}

\newcommand\labvxleft[4][black]{% #1 optional colour, #2 vertex name, #3 label name, #4 label
  \labelsolo {#3} {#4}
  \vxrightsep[#1] {#2} {#3} \vertextoitslabelgap
}

\newcommand\singletonright[2][black]{\begin{tikzpicture}[baseline]\labvxright[#1]{v}{l}{#2}\end{tikzpicture}}
\newcommand\singletonleft[2][black]{\begin{tikzpicture}[baseline]\labvxleft[#1]{v}{l}{#2}\end{tikzpicture}}

\newcommand\singleton[2][black]{\singletonright[#1]{#2}}

\newcommand\singletonx{\singleton x}
\newcommand\singletony{\singleton y}
\newcommand\singletonz{\singleton z}
\newcommand\singletonp{\singleton \p}
\newcommand\singletonq{\singleton \q}
\newcommand\singletonqq{\singleton \qq}
\newcommand\singletonpx{\singleton \px}
\newcommand\singletonpy{\singleton \py}
\newcommand\singletonppy{\singleton \ppy}
\newcommand\singletonqx{\singleton \qx}
\newcommand\singletonrx{\singleton \rx}
\newcommand\singletonqy{\singleton \qy}
\newcommand\singletonpxy{\singleton \pxy}
\newcommand\singletonppx{\singleton \ppx}

%% Drawing pairs of vertices inline, labelled/coloured, with possible edge

\newcommand\fullynamedcolouredpair[8]{% #1 left label name, #2 left label, #3 left vx name, #4 left colour, #5 right vx name, #6 right vx colour, #7 right label name, #8 right label
\labelsolo {#1} {#2}
\vxright[#4] {#3} {#1}
\vxrightsep[#6] {#5} {#3} {\edgepairvxgap}
\labelright {#7} {#8} {#5}
\e {#3} {#5}
}
% \newcommand\fullynamedcolouredpair[8]{% #1 left label name, #2 left label, #3 left vx name, #4 left colour, #5 right vx name, #6 right vx colour, #7 right label name, #8 right label
% \labelsolo {#1} {#2}
% \vxright[#4] {#3} {#1}
% \vxrightsep[#6] {#5} {#3} {\edgepairvxgap}
% \labelright {#7} {#8} {#5}
% \e {#3} {#5}
% }
\newcommand\fullynamedpair[6]{% #1 left label name, #2 left label, #3 left vx name, #4 right vx name, #5 right label name, #6 right label
  \fullynamedcolouredpair{#1}{#2}{#3}{black}{#4}{black}{#5}{#6}}
\newcommand\namedpair[4]{% #1 left label, #2 left vx name, #3 right vx name, #4 right label
  \fullynamedpair{leftlabel}{#1}{#2}{#3}{rightlabel}{#4}}
\newcommand\colouredpair[4]{% #1 left label, #2 left vx colour, #3 right vx colour, #4 right label
  \fullynamedcolouredpair{leftlabel}{#1}{leftvx}{#2}{rightvx}{#3}{rightlabel}{#4}}
%\newcommand\colouredpairafter[5]{% #1 left label, #2 left vx colour, #3 right vx colour, #4 right label, #5 node name right of left label
%  \fullynamedcolouredpairafter{leftlabel}{#1}{leftvx}{#2}{rightvx}{#3}{rightlabel}{#4}{#5}}
\newcommand\edgepair[2]{% #1 left label, #2 right label
  \namedpair{#1}{leftvx}{rightvx}{#2}}
\newcommand\edgepairpic[2]{% #1 left label, #2 right label
\begin{tikzpicture}[inlinegraph]\edgepair{#1}{#2}\end{tikzpicture}}
\newcommand\colouredpairpic[4]{% #1 left label, #2 left vx colour, #3 right vx colour, #4 right label
\begin{tikzpicture}[inlinegraph]\colouredpair{#1}{#2}{#3}{#4}\end{tikzpicture}}

\newcommand\unlabelledcolouredpairpic[2]{\colouredpairpic{}{#1}{#2}{}}

\newcommand\unlabelledsingleton[1][black]{\begin{tikzpicture}[baseline]
	\path[draw=none] (0,0) -- (0,1ex);  % [baseline] plus fake path => center is same as 'x' center, vertically
	\node[#1 vx, anchor=center] at (0,.5ex) {};
\end{tikzpicture}} 
\newcommand\unlabelledbluesingleton{\unlabelledsingleton[blue]}




%%%%%%%%% EDGES

%%% Edge styles

\newcommand\defaultarrowstyle{stealth'}% all directed edges, e.g. binding edges
\tikzset{>=\defaultarrowstyle}

\newcommand\linewd{.5pt}
\newcommand\linewidthforbindinggraphsymbol{.4pt}
\newcommand\fibrewidth{1pt}
\newcommand\leapwidth{.7pt}
\newcommand\dualitywidth{.5pt}

%%% Edges of various types

% hack to stretch the dash pattern by a factor:
\makeatletter
\tikzset{%
  stretch dash/.code args={on #1 off #2}{%
    \tikz@addoption{%
      \pgfgetpath\currentpath%
      \pgfprocessround{\currentpath}{\currentpath}%
      \pgf@decorate@parsesoftpath{\currentpath}{\currentpath}%
      \pgfmathparse{max(round((\pgf@decorate@totalpathlength-#1)/(#1+#2)),0)}%
      \let\npattern=\pgfmathresult%
      \pgfmathparse{\pgf@decorate@totalpathlength/(\npattern*(#1+#2)+#1)}%
      \let\spattern=\pgfmathresult%
      \pgfsetdash{{\spattern*#1}{\spattern*#2}}{0pt}%
    }%
  }%
}
\makeatother


% Undirected edge (no separation from source and target --- touches them)
\newcommand\e[2]{\draw ({#1})--({#2});}

% Undirected edge with extra endpoint separations
\newcommand\eseps[4]{\draw[{_[sep=#3]}-{_[sep=#4]}] (#1) -- (#2);}

% Fibration edge (NB assumes ambient style for dotting; this only sets end-point gaps)
\newcommand\fe[2]{\draw ([yshift=-2.7pt]#1.south)--([yshift=2.7pt]#2.north);} 

% Directed edge with extra endpoint separations
\newcommand\deseps[5][\defaultarrowstyle]{\draw[{_[sep=#4]}-{>[sep=#5]},>=#1] (#2) -- (#3);}% #1 optional arrow style. For arrowhead adjustment using arrows.meta: https://tex.stackexchange.com/questions/150721/adjusting-the-size-of-an-arrow/150739#150739

% Directed edge (default slightly separated from source and target)
\newcommand\de[2]{\deseps{#1}{#2}{2.4pt}{2.3pt}}

% For much better dashed/dotted liens, pstricks-style
\makeatletter





%%%%%% GRAPHS AND TREES

% standard edge lengths sizes
\def\edgelen{.8}% basic edge of graph
\SQUAREROOT{.75}{\equitriangleheightmultiplier}
\MULTIPLY{\edgelen}{\equitriangleheightmultiplier}{\height}% height of
                                % equilateral of base \edgelen
\DIVIDE{\edgelen}{2}{\halfedgelen}
\DIVIDE{\halfedgelen}{2}{\quarteredgelen}
\DIVIDE{\quarteredgelen}{2}{\eighthedgelen}
\DIVIDE{\height}{2}{\halfheight}
\MULTIPLY{\edgelen}{2}{\twoedgelen}
\MULTIPLY{\edgelen}{3}{\threeedgelen}


\tikzstyle{tree} = [graph, level distance=4ex, outer sep=0pt, inner sep=0, 
  level 1/.style={sibling distance=7ex},
  level 2/.style={sibling distance=4ex},
  level 3/.style={sibling distance=4ex}
]
\tikzstyle{leaf} = [black vx, outer sep=2pt]

\newcommand\gr[1]{\vctr{\begin{tikzpicture}[graph]#1\end{tikzpicture}}}
\newcommand\tr[1]{\vctr{\begin{tikzpicture}[tree,
  level 1/.style={sibling distance=5ex},
  level 2/.style={sibling distance=5ex},
  level 3/.style={sibling distance=5ex}
]#1\end{tikzpicture}}}

\newcommand\namedleaf[1]{ node[leaf] (#1) {} }%
\newcommand\leafsep[1]{ node[leaf]{} edge from parent[{_[sep=#1]}-] }% leaf whose edge from parent takes extra sep at parent
\newcommand\leaf{ node[leaf]{} } % edge from parent[{_[sep=#1]}]}
\newcommand\namedleaflab[2] {  node[leaf, label={[label distance=1pt]{-90}:${#1}$}] (#2) {} } 
\newcommand\leaflab[1] {\namedleaflab{#1}{leaf}}
\newcommand\leaflabsep[2] {% #1 label, #2 extra separation from parent
node[leaf, label={[label distance=1pt]{-90}:${#1}$}]{} edge from parent[{_[sep=#2]}-]
}
\newcommand\unionroot{\node[circle] {$\graphunion$}}% for some reason "\node" only works at the root
\newcommand\joinroot{\node[circle] {$\graphjoin$}}
\newcommand\unionnode{node[circle] {$\graphunion$}}
\newcommand\joinnode{node[circle] {$\graphjoin$}}
\newcommand\namedunionnode[1]{node[circle] (#1) {$\graphunion$}}
\newcommand\namedjoinnode[1]{node[circle] (#1) {$\graphjoin$}}
\newcommand\dummyleaf{ edge from parent[draw=none] }



%%%%%% FIBRATIONS


\tikzstyle{graph}        = [line width=\linewd]
\tikzstyle{fibres}       = 
 [every path/.style={graph, line cap=round, line width=\fibrewidth, stretch dash=on .001pt off 4.2pt}]
\tikzstyle{leap}         = [every path/.style={graph, line width=\leapwidth, stretch dash=on 5pt off 3pt}]
\tikzstyle{duality}      = 
 [every path/.style=
  {graph, line width=\dualitywidth, color=dualitycolour, stretch dash=on 3pt off 1.5pt}]
\tikzstyle{binding}      = [every path/.style={graph, line width=\dualitywidth, color=bindingcolour}]
\tikzstyle{semifibres}   = [graph]
\tikzstyle{inlinegraph} = [graph,baseline]


\newcommand\fibheight{1.6}



%%% Condensed "inline" skew fibrations / cps

% standard inline vertex heights
\def\lo{.13}%
\def\mi{.23}%
\def\hi{.33}%
\def\vi{.53}% "very hi"

\tikzset{inlinecp/.style={graph,auto,inner sep=0pt,outer sep=0pt,node distance = 0pt}}
\tikzset{% for inline skew fibrations
  token/.style={
	text height=1.9ex,
	text depth =.5ex,
	inner sep=0pt,
	outer sep=0pt
  }
}
\newcommand\fibration[1]{% the fibres to lay draw in succession
\begin{math}\begin{scope}[inlinecp,start chain=going right]#1\end{scope}\end{math}}
\newcommand\fibrationpic[1]{\begin{tikzpicture}[graph]\fibration{#1}\end{tikzpicture}}
\newcommand\fibbase[2]{% name, contents
\node[on chain,token] (#1) {${}#2{}$};}
\renewcommand\over[3]{\node[#1,above = #2 of \currentnode, anchor=center] (#3) {};}% style, height, name
\newcommand\overlabdirgap[6]{% style, height, name, label, label-direction, label-gap
\node[#1,above = #2 of \currentnode, anchor=center, label={[outer sep=0pt,inner sep=0pt,label distance={#6}]{#5}:${#4}$}] ({#3}) {};}
\newcommand\overlabdir[5]{% style, height, name, label, label-direction (default label distance)
\overlabdirgap{#1}{#2}{#3}{#4}{#5}{\lvxugap}}
\newcommand\overlabu[4]{% label up: style, height, name, label
\overlabdir{#1}{#2}{#3}{#4}{90}}
\newcommand\overlabd[4]{% label down: style, height, name, label
\overlabdir{#1}{#2}{#3}{#4}{-90}}
\newcommand\overlabdd[4]{% label far down: style, height, name, label
\overlabdirgap{#1}{#2}{#3}{#4}{-90}{\lvxddgap}}
\newcommand\fibre[3]{% name, base, fibres
\fibbase{#1}{#2}\def\currentnode{#1}#3}
\newcommand\symb[1]{\fibbase{symb}{#1}}
\newcommand\open{\symb{(}}
\newcommand\close{\symb{)}}



%%%%%%%%%%% DRINKER EXAMPLES


\newcommand\drinkerformula{\exists x\mkern1mu(\mkern1mu px\mkern-1mu\implies\mkern-1mu \aypy)}
\newcommand\veedrinkerformula{\exists x\mkern1mu(\mkern1mu\ppx\mkern-1mu\vee\mkern-1mu \aypy)}
\newcommand\variantveedrinkerformula{\exists x\mkern2mu\forall y\mkern2mu(\py\mkern-1mu\vee\mkern-1mu\ppx)}

\newcommand\drinkergraph{\gD}

% Parameters for drinker fibration

\newcommand\drinkerx{-1.1}
\newcommand\drinkerxx{-.4}% -.4
\newcommand\drinkerxxx{.4}% .4
\newcommand\drinkerxxxx{1.15}
%\MULTIPLY{\drinkerx}{2}{\drinkerxx}
%\MULTIPLY{\drinkerx}{3}{\drinkerxxx}
\newcommand\drinkerup{.27}
\newcommand\drinkerdown{.235}
\newcommand\drinkerdowndown{.37}
\newcommand\drinkercovergap{.445}
\newcommand\drinkercoverfirstdown{.3}
\newcommand\drinkercoverseconddown{.21}
% \newcommand\drinkercovergap{.45}
\newcommand\drinkercoverup{.24}
\newcommand\drinkercoverdown{.27}


%%%% Graph of drinker

\newcommand\drinkerbasevertices{
\lvxd{\drinkerx,0}{x}{x}
\lvxd{\drinkerxx,-\drinkerdowndown}{ppx}{\ppx}%{\ppxlabelangle}
\lvxd{\drinkerxxx,-\drinkerdown}{y}{y}%{\ylabelangle}
\lvxd{\drinkerxxxx,0}{py}{\py}
}

\newcommand\drinkerbaseedges{
\begin{scope}[graph]
\e x y
\e x {py}
\e x {ppx}
\end{scope}
}

\newcommand\drinkerbase{
\drinkerbasevertices
\drinkerbaseedges
}


%%%% CP of drinker

\newcommand\drinkercoververticescoloured{
\begin{scope}[shift={(0,\drinkercovergap)}]
    \vx{\drinkerx,0}{xa}
    \vx{\drinkerxxx,-\drinkercoverseconddown}{ya}
    \bluevx{\drinkerxxxx,0}{pya}
  \end{scope}
  \vx{\drinkerx,0}{xb}
  \bluevx{\drinkerxx,-\drinkercoverfirstdown}{ppxb}
}

\newcommand\drinkercoveredges{
\begin{scope}[graph]
  \e{xb}{ppxb}
  \e{xa}{pya}
  \e{xa}{ya}
\end{scope}}


\newcommand\drinkercovercoloured{
  \drinkercoververticescoloured
  \drinkercoveredges
}

\newcommand\drinkerfibres{
\begin{scope}[fibres]
  \fe{xa}{xb}
  \fe{xb}{x}
  \fe{ppxb}{ppx}
  \fe{pya}{py}
  \fe{ya}{y}
\end{scope}
}

\newcommand\drinkerfibcoloured{
\begin{scope}[shift={(0,\fibheight)}]
  \drinkercovercoloured
\end{scope}
\drinkerbase
\drinkerfibres
}

% Inline versions of drinker CP

\newcommand\drinkerInlineAbstract[4]{\fibration{
  \fibre{e} {#1}{
    \over{black vx}{\lo}{el}
    \over{black vx}{\hi}{eh}
  }
  \symb{\mkern2mu(\mkern2mu}
  \fibre{px} {#2}{
    \over{blue vx}{\lo}{pxl}
  }
  \symb{\mkern-2mu\implies\mkern-2mu}
  \fibre{a} {#3}{
    \over{black vx}{\lo}{al}
  }
  \symb{\mkern3mu}
  \fibre{py} {#4}{
    \over{blue vx}{\lo}{pyl}
  }
  \symb{\mkern2mu)}
  \draw (pxl) -- (el);
  \draw (eh) to[out=-2,in=162,looseness=.9] (al);
  \draw (eh) to[out=6,in=155,looseness=.9] (pyl);
}}

\newcommand\veedrinkerinlinevee{\mkern-1mu\vee\mkern-0mu}
\newcommand\veedrinkerinlineveesymb{\symb{\veedrinkerinlinevee}}
\newcommand\veedrinkerInlineAbstract[4]{\fibration{
  \fibre{e} {#1}{
    \over{black vx}{\lo}{el}
    \over{black vx}{\hi}{eh}
  }
  \symb{\mkern2mu(\mkern2mu}
  \fibre{px} {#2}{
    \over{blue vx}{\lo}{pxl}
  }
  \veedrinkerinlineveesymb
  \fibre{a} {#3}{
    \over{black vx}{\lo}{al}
  }
  \symb{\mkern3mu}
  \fibre{py} {#4}{
    \over{blue vx}{\lo}{pyl}
  }
  \symb{\mkern2mu)}
  \draw (pxl) -- (el);
  \draw (eh) to[out=-2,in=162,looseness=.9] (al);
  \draw (eh) to[out=6,in=155,looseness=.9] (pyl);
}}

\newcommand\drinkerInline{\drinkerInlineAbstract{\exists x}{\px}{\forall y}{\py}}

\newcommand\veedrinkerInline{\veedrinkerInlineAbstract{\exists x}{\ppx}{\forall y}{\py}}


%%%%%%%%%%%%%%%%%%%%%%% FIGURES

% rule underneath caption
\newcommand\figrule{\vspace{1.3ex}\hrule}%\vspace{2ex}}

%%%%%%%% FIGURE: Four combinatorial proofs

\newcommand\pfyxone{-1.2}
\newcommand\pfyxtwo{-.6}
\newcommand\pfyxthree{0}
\newcommand\pfyxfour{.6}
\newcommand\pfyxfive{1.2}
\newcommand\pfycoverradius{.23}
\newcommand\pfycp{
  \begin{scope}[shift={(0,\fibheight)}]
    % cover vertices (coloured)
    \begin{scope}[shift={(0,-\pfycoverradius)}]
      \vx{\pfyxone,0}{xa}
      \bluevx{\pfyxtwo,0}{ppxa}
    \end{scope}
    \begin{scope}[shift={(0,\pfycoverradius)}]
      \vx{\pfyxone,0}{xb}
      \redvx{\pfyxtwo,0}{ppxb}
    \end{scope}
    \vx{\pfyxthree,0}{yc}
    \bluevx{\pfyxfour,0}{pyc}
    \redvx{\pfyxfive,0}{pfyc}
  \end{scope}
  % % base vertices
  \lvxd{\pfyxone,0} x {\likepbar x}
  \lvxd{\pfyxtwo,0}{ppx}{\likepbar\ppx}
  \lvxd{\pfyxthree,0}{y}{\likepbar y}
  \lvxd{\pfyxfour,0}{py}{\likepbar\py}
  \lvxd{\pfyxfive,0}{pfy}{\mkern14mu\likepbar\pfy}
  % cover edges
  \e {xa} {ppxa}
  \e {xb} {ppxb}
  \e {pyc} {pfyc}
  % base edges
  \e {x} {ppx}
  \e {py} {pfy}
  % fibration edges
  \begin{scope}[fibres]
    \fe{xb}{xa}
    \fe{xa}{x}
    \fe{ppxb}{ppxa}
    \fe{ppxa}{ppx}
    \fe{yc}{y}
    \fe{pfyc}{pfy}
    \fe{pyc}{py}
  \end{scope}
}


\newcommand\pabxone{-1.2}
\newcommand\pabxtwo{-.5}
\newcommand\pabxthree{.2}
\newcommand\pabxfour{.7}
\newcommand\pabxfive{1.2}
\newcommand\pabcoverradius{.3}% .23 gives two dots instead of three
\newcommand\pabnudge{-.22}% for triangles
\newcommand\pabcp{
  \begin{scope}[shift={(0,\fibheight)}]
    % cover vertices (coloured)
    \bluevx{\pabxone,0}{pabm}
    \redvx{\pabxtwo,0}{pcdm}
    \begin{scope}[shift={(0,\pabcoverradius)}]
      \vx{\pabxthree,0}{xu}
      \vx{\pabxfour,\pabnudge}{yu}
      \bluevx{\pabxfive,0}{pxyu}
    \end{scope}
    \begin{scope}[shift={(0,-\pabcoverradius)}]     
      \vx{\pabxthree,0}{xl}
      \vx{\pabxfour,\pabnudge}{yl}
      \redvx{\pabxfive,0}{pxyl}
    \end{scope}
  \end{scope}
  % base vertices
  \lvxd{\pabxone,0}{pab}{\hspace{-1ex}\likepbar\qqab}
  \lvxd{\pabxtwo,0}{pcd}{\likepbar\qqba\hspace{-.5ex}}
  \lvxd{\pabxthree,0}{x}{\likepbar x} 
  \lvxd{\pabxfour,\pabnudge}{y}{y}
  \lvxd{\pabxfive,0}{pxy}{\likepbar\qxy\hspace{-1ex}}
  % cover edges
  \e {pabm} {pcdm}
  \e {xu} {yu}
  \e {xu} {pxyu}
  \e {yu} {pxyu}
  \e {xl} {yl}
  \e {xl} {pxyl}
  \e {yl} {pxyl}
  % base edges
  \e {pab} {pcd}
  \e {x} {y}
  \e {x} {pxy}
  \e {y} {pxy}
  % fibration edges
  \begin{scope}[fibres]
    \fe{pabm}{pab}
    \fe{pcdm}{pcd}
    \fe{xu}{xl}
    \fe{xl}{x}
    \fe{yu}{yl}
    \fe{yl}{y}
    \fe{pxyu}{pxyl}
    \fe{pxyl}{pxy}
  \end{scope}
}
\newcommand\pabcpinline{\fibration{
  \def\hi{0.43}
  \fibre{qabbase}{\qab}{
    \over{blue vx}{\lo}{pab}
  }
  \symb{\vee}
  \fibre{qbabase}{\qba}{
    \over{red vx}{\lo}{pcd}
  }
  \symb{\mkern5mu\implies\mkern4mu}
  \fibre{exbase}{\exists x}{
    \over{black vx}{\lo}{xl}
    \over{black vx}{\hi}{xu}
  }
  \symb{\mkern4mu}
  \fibre{eybase}{\exists y}{
    \over{black vx}{\lo}{yl}
    \over{black vx}{\hi}{yu}
  }
  \symb{\mkern3mu}
  \fibre{qxybase}{\qxy}{
    \over{red vx}{\lo}{pxyl}
    \over{blue vx}{\hi}{pxyu}
  }
  \e {pab} {pcd}
  \e {xl} {yl}
  \e {yl} {pxyl}
  \e {xu} {yu}
  \e {yu} {pxyu}
  \draw (xl) to[out=25,in=160,looseness=.9](pxyl);
  \draw (xu) to[out=25,in=160,looseness=.9](pxyu);
}}

\newcommand\onionxone{.3}
\newcommand\onionxtwo{.85}
\newcommand\onionxthree{1.4}
\newcommand\onionxfour{2.1}
\newcommand\onionxfive{2.8}
\newcommand\onionxsix{3.5}
\newcommand\onioncoverradius{.3}
\newcommand\onionnudge{-.22}% for triangles
\newcommand\onioncp{
  \begin{scope}[shift={(0,\fibheight)}]
    % cover vertices (coloured)
    \begin{scope}[shift={(0,-\onioncoverradius)}]
      \vx{\onionxone,0}{xa}
      \redvx{\onionxtwo,\onionnudge}{pfxa}
      \greenvx{\onionxthree,0}{ppxa}
    \end{scope}
    \begin{scope}[shift={(0,\onioncoverradius)}]
      \vx{\onionxone,0}{xb}
      \greenvx{\onionxtwo,\onionnudge}{pfxb}
      \bluevx{\onionxthree,0}{ppxb}
    \end{scope}
    \vx{\onionxfour,0}{yc}
    \redvx{\onionxfive,0}{ppffyc}
    \bluevx{\onionxsix,0}{pyc}
  \end{scope}
  % base vertices
  \lvxd{\onionxone,0} x {\likepbar x\hspace{1ex}}
  \lvxd{\onionxtwo,\onionnudge}{pfx}{\pfx}
  \lvxd{\onionxthree,0}{ppx}{\hspace{1.5ex}\ppx}
  \lvxd{\onionxfour,0} y {\likepbar y}
  \lvxd{\onionxfive,0}{ppffy}{\likepbar\ppffy\hspace{.2ex}}
  \lvxd{\onionxsix,0}{py}{\hspace{.5ex}\likepbar\py}
  % cover edges
  \e {xa} {pfxa}
  \e {xa} {ppxa}
  \e {pfxa} {ppxa}
  \e {xb} {pfxb}
  \e {xb} {ppxb}
  \e {pfxb} {ppxb}
  % base edges
  \e {x} {pfx}
  \e {x} {ppx}
  \e {pfx} {ppx}
  % fibration edges
  \begin{scope}[fibres]
    \fe{xb}{xa}
    \fe {xa}{x}
    \fe {pfxb}{pfxa}
    \fe {pfxa}{pfx}
    \fe {ppxb}{ppxa}
    \fe {ppxa}{ppx}
    \fe {yc}{y}
    \fe {ppffyc}{ppffy}
    \fe {pyc}{py}
  \end{scope}
}
\newcommand\onioncpinline{\fibration{
  \def\hi{.43}
  \symb{\left(\strut\mkern-5mu\right.}
  \fibre{axbase}{\allx}{
    \over{black vx}{\lo}{xl}
    \over{black vx}{\hi}{xu}
  }
  \symb{(}
  \fibre{pfxbase}{\pfx}{
    \over{red vx}{\lo}{pfxl}
    \over{green vx}{\hi}{pfxu}
  }
  \symb{\mkern-2mu\implies\mkern-2mu}
  \fibre{pxbase}{\px}{
    \over{green vx}{\lo}{pxl}
    \over{blue vx}{\hi}{pxu}
  }
  \symb{)\left.\mkern-5mu\strut\right)\mkern3mu\implies\mkern3mu}
  \fibre{aybase}{\ally}{
    \over{black vx}{\lo}{y}
  }
  \symb{\mkern2mu(\mkern1mu}
  \fibre{pffybase}{\pffy}{
    \over{red vx}{\lo}{pffy}
  }
  \symb{\mkern-1mu\implies\mkern-1mu}
  \fibre{pybase}{\py}{
    \over{blue vx}{\lo}{py}
  }
  \symb{\mkern1mu)}
%
  \e {xl} {pfxl}
  \e {pfxl} {pxl}
  \e {xu} {pfxu}
  \e {pfxu} {pxu}
  \draw (xu) to[out=23,in=170,looseness=.9](pxu);
  \draw (xl) to[out=23,in=170,looseness=.9](pxl);
}}

\newcommand\eabxone{-.8}
\newcommand\eabxtwo{-.3}
\newcommand\eabxthree{.3}
\newcommand\eabxfour{.825}
%\newcommand\eabcoverradius{.44}%{.3}
\newcommand\eabcoverradius{.22}%{.3}
\newcommand\eaby{-.2}
\newcommand\eabyy{-.35}
\newcommand\eabfibheightboost{.12}
\newcommand\eabcp{
  \begin{scope}[shift={(0,\fibheight)}]
    \begin{scope}[shift={(0,\eabfibheightboost)}]
      % cover vertices (coloured)
      \begin{scope}[shift={(0,-\eabcoverradius)}]
        % cover bottom triangle
        \vx{\eabxone,0}{xl}
        \bluevx{\eabxtwo,\eabyy}{ppal}
        \redvx{\eabxthree,\eaby}{ppbl}
        \redvx{\eabxfour,0}{pxm}
      \end{scope}
      % cover mid horizontal
%      \vx{\eabxone,0}{xm}
%      \redvx{\eabxfour,0}{pxm}
      \begin{scope}[shift={(0,\eabcoverradius)}]
        % cover upper horizontal
        \vx{\eabxone,0}{xu}
        \bluevx{\eabxfour,0}{pxu}
      \end{scope}
    \end{scope}
  \end{scope}
  % base vertices
  \lvxd{\eabxone,0} x {x\hspace{1.3ex}}
  \lvxd{\eabxtwo,\eabyy}{ppa}{\likepbar\ppa\hspace{.2ex}}
  \lvxd{\eabxthree,\eaby}{ppb}{\hspace{.9ex}\likepbar\ppb}
  \lvxd{\eabxfour,0}{px}{\px\hspace{-1.5ex}}
  % cover edges
  \e {xl} {ppal}
  \e {xl} {ppbl}
  \e {ppal} {ppbl}
  \e {xl} {pxm}
%  \e {xm} {pxm}
  \e {xu} {pxu}
  % base edges
  \e {x} {ppa}
  \e {x} {ppb}
  \e {ppa} {ppb}
  \e {x} {px}
  % fibration edges
  \begin{scope}[fibres]
    \fe {xu}{xl}
%    \fe {xu}{xm}
%    \fe {xm}{xl}
    \fe {xl}{x}
    \fe {ppal}{ppa}
    \fe {ppbl}{ppb}
    \fe {pxu}{pxm}
    \fe {pxm}{px}
  \end{scope}
}




\newcommand\pfyformula{(\axpx) \mkern1mu \implies \mkern1mu \forall y\mkern3mu (\mkern1mu\py\tightwedge \pfy\mkern1mu)}
\newcommand\eabformula{\exists x\mkern2mu (\mkern1mu\pa\tightvee\mkern1.2mu\pb\mkern-1mu\implies\mkern-1mu px\mkern1mu)}
\newcommand\onionformula{\left(\strut\forall x\mkern1mu  (\mkern1mu \pfx\tightimplies \px)\right)\mkern2mu \implies\mkern2mu  \forall y\mkern2mu  (\mkern1mu \pffy\tightimplies \py\mkern1mu )}
\newcommand\pabformula{\qab\vee \qba\mkern4mu \implies \mkern4mu \exists x\mkern4mu \exists y\mkern3mu \qxy}

\newcommand\cponeformula{\eabformula}
\newcommand\cptwoformula{\pfyformula}
\newcommand\cpthreeformula{\pabformula}
\newcommand\cpfourformula{\onionformula}

\newcommand\cpone{\eabcp}
\newcommand\cptwo{\pfycp}
\newcommand\cpthree{\pabcp}
\newcommand\cpfour{\onioncp}

\newcommand\cponformula[3]{% #1 cp, #2 xadjustment, #3 formula
  \begin{scope}[graph,shift={(#2,1.7)}]{#1}\end{scope}
  \node[anchor=base] (formula) at (0,0){$#3$};
}

\newcommand\eabcponformula{\cponformula{\eabcp}{-.1}{\eabformula}}
\newcommand\pfycponformula{\cponformula{\pfycp}{-.15}{\pfyformula}}
\newcommand\pabcponformula{\cponformula{\pabcp}{0}{\pabformula}}
\newcommand\onioncponformula{\cponformula{\onioncp}{-2}{\onionformula}}

\newcommand\cponeonformula{\eabcponformula}
\newcommand\cptwoonformula{\pfycponformula}
\newcommand\cpthreeonformula{\pabcponformula}
\newcommand\cpfouronformula{\onioncponformula}


\newcommand\figcps{\begin{figure*}\begin{center}\vspace{1ex}\begin{tikzpicture}\newcommand\radius{3.6}%
\begin{scope}[shift={(-\radius,5.7)}]{\cponeonformula}\end{scope}
\begin{scope}[shift={(\radius,5.7)}]{\cptwoonformula}\end{scope}
\begin{scope}[shift={(-\radius,0)}]{\cpthreeonformula}\end{scope}
\begin{scope}[shift={(\radius,0)}]{\cpfouronformula}\end{scope}
\end{tikzpicture}\end{center}%
\caption{\label{fig:cps}%
  Four combinatorial proofs, each shown above the formula proved.
  Here $x$ and $y$ are variables, $f$ is a unary function symbol, $a$ and $b$ are constants (nullary function symbols), $p$ is a unary predicate symbol, and $q$ is a binary predicate symbol.}\figrule\end{figure*}}






%%%%%%%%%% FIGURE Condensed forms of the four combinatorial proofs

\newcommand\eabcpinline{\fibration{
  \def\hi{.39}
  \def\vi{.66}
  \fibre{exbase}{\existsx}{
    \over{black vx}{\lo}{x}
    \over{black vx}{\hi}{xm}
    \over{black vx}{\vi}{xu}
  }
  \symb{\mkern2mu(\mkern2mu}
  \fibre{pabase}{\pa}{
    \over{blue vx}{\lo}{pa}
  }
  \symb{\tightvee\mkern1.2mu}
  \fibre{pybase}{\pb}{
    \over{red vx}{\lo}{pb}
  }
  \symb{\mkern-1mu\implies\mkern-1mu}
  \fibre{pxbase}{\px}{
    \over{red vx}{\lo}{px}
    \over{blue vx}{\hi}{pxm}
  }
  \symb{\mkern2mu)}
  \e {x}{pa}
  \e {pa}{pb}
  \draw (x) to[out=20,in=165,looseness=.85] (pb);
  \draw (xm) to[out=20,in=163,looseness=.8] (px);
  \draw (xu) to[out=20,in=163,looseness=.8] (pxm);
}}

\newcommand\pfycpinline{\fibration{
  \symb{(}
  \fibre{axbase}{\allx}{
     \over{black vx}{\lo}{xl}
     \over{black vx}{\hi}{xu}
  }
  \symb{\mkern2mu}
  \fibre{pxbase}{\px}{
    \over{blue vx}{\lo}{pxl}
    \over{red vx}{\hi}{pxu}
  }
  \symb{\mkern2mu)\mkern4mu\implies\mkern4mu}
  \fibre{aybase}{\ally}{
    \over{black vx}{\lo}{y}
  }
  \symb{\mkern2mu(\mkern1mu}
  \fibre{pybase}{\py}{
    \over{blue vx}{\lo}{py}
  }
  \symb{\mkern-2mu\wedge\mkern-2mu}
  \fibre{pfybase}{\pfy}{
    \over{red vx}{\lo}{pfy}
  }
  \symb{\mkern1mu)}
  \draw (xu) -- (pxu);
  \draw (xl) -- (pxl);
  \draw (py) -- (pfy);
}}


\newcommand\cponeinline{\eabcpinline}
\newcommand\cptwoinline{\pfycpinline}
\newcommand\cpthreeinline{\pabcpinline}
\newcommand\cpfourinline{\onioncpinline}



\newcommand\figcpscondensed{\begin{figure*}\begin{center}\vspace{-1ex}\begin{tikzpicture}\newcommand\radius{3.2}%
  \begin{scope}[shift={(-\radius,1.9)}]\cponeinline\end{scope}
  \begin{scope}[shift={(\radius,1.9)}]\cptwoinline\end{scope}
  \begin{scope}[shift={(-\radius,0)}]\cpthreeinline\end{scope}
  \begin{scope}[shift={(\radius,0)}]\cpfourinline\end{scope}
\end{tikzpicture}\end{center}%
\caption{\label{fig:cps-condensed}Condensed forms of the four combinatorial proofs in Fig.\,\ref{fig:cps}.}\figrule\vspace{-4ex}\end{figure*}}




%%%%%%% PICTURE: Drinker graph, cotree, binding graph


\newcommand\drinkersquare{
\lvx{-\halfedgelen,\halfedgelen}{x}{x}{135}
\lvx{\halfedgelen,\halfedgelen}{px}{\mkern.5mu\likex\ppx}{45}
\lvx{-\halfedgelen,-\halfedgelen}{y}{\mkern.5mu\likex y}{-135}
\lvx{\halfedgelen,-\halfedgelen}{py}{\py}{-45}
}
\newcommand\drinkersquareedges{
  \e x y
  \e x {px}
  \e x {py}
}

\newcommand\drinkersquarewithedges{\drinkersquare\drinkersquareedges}


\newcommand\drinkergraphpic{\gr{\drinkersquarewithedges}}

\newcommand\drinkercotreepic{\tr{\tikzset{level 1/.style={sibling distance=8ex},level 2/.style={sibling distance=4ex}}
  \joinroot
    child { \leaflab x }
    child { 
      \unionnode
      child { \leaflab y }
      child { \leaflab \ppx }
      child { \leaflab \py }
    }
  ;
}}

\newcommand\drinkerbindinggraphpic{\gr{\drinkersquare \de x {px} \de y {py}}}


%%%%%% PICTURE lifting diagrams

\newcommand\liftingdiagrams{%
\begin{center}\begin{tikzpicture}[graph,outer sep=1.5pt,inner sep=0pt]\begin{math}
  \newcommand\rad{.8}
  \newcommand\vht{1.55}
  \newcommand\shortrad{.4}
  \newcommand\commonsquarepart{
    \node (liftw) at (-\rad,\vht) {$\skewlifting\vertexa$};
    \node[left=-1pt of liftw,outer sep=0,inner sep=0] (uniqueliftw) {$\exists\mkern1mu!$};
    \node (w) at (-\rad,0) {$\vertexa$};
    \node (v) at (\rad,1.55) {$v$};
    \node (fv) at (\rad,0) {$\fib(v)$};
    % fibres
    \begin{scope}[fibres]
      \fe{liftw}{w}
      \fe{v}{fv}
    \end{scope}
  }
  \begin{scope}[shift={(-3,0)}]    
    \commonsquarepart
    % undirected edges
    \eseps {w} {fv} {1.7pt} {1.7pt}
    \eseps {liftw} {v} {1.7pt} {1.7pt}
  \end{scope}
  \begin{scope}
    \node (v) at (\rad,1.55) {$v$};
    \node (fv) at (\rad,0) {$\fib(v)$};
    \node (fliftw) at (-\rad,.3){$\fib(\skewlifting w)$};
    \node (liftw) at (-\rad,1.8){$\skewlifting\vertexa$};
    \node[left=-1pt of liftw,outer sep=0,inner sep=0] (uniqueliftw) {$\exists$};
    \node (w) at (-\shortrad,-.35) {$w$};
    % undirected edges
    \eseps {w} {fv} {1.7pt} {1.7pt}
    \eseps {liftw} {v} {1.7pt} {1.7pt}
    \eseps {fliftw} {fv} {.3pt} {1.7pt}
  \end{scope}
  \begin{scope}[fibres]
    \fe{liftw}{fliftw}
    \fe{v}{fv}
  \end{scope}
  \begin{scope}[shift={(3,0)}]    
    \commonsquarepart
    % directed edges
    \deseps {w} {fv} {1.8pt} {1.5pt}
    \deseps {liftw} {v} {1.8pt} {1.5pt}
  \end{scope}
\end{math}\end{tikzpicture}\end{center}}




%%%%%%% FIGURE: Drinker bifibration, binding fibration, and skeleton

% add a label to a vertex
% relative positioning using polar coordinates https://www.latex4technics.com/?note=30vp
\tikzset{
    position/.style args={#1:#2 from #3}{
        at=(#3.#1), anchor=#1+180, shift=(#1:#2)
    }
}
\newcommand\labsep[4]{% #1 point-being-labelled, #2 contents, #3 angle, #4 labelsep
%\node (label) at ([shift={(#1)}] #3:#4) {\ensuremath{#2}};}
\node[outer sep=0,inner sep=0,position=#3:#4 from #1] {\ensuremath{#2}};}

\newcommand\coverylabelangle{-35}
\newcommand\coverppxlabelangle{0}
\newcommand\coverppxlabeldist{2pt}
\newcommand\addalldrinkercoverlabels[3]{% #1 = upper x variant, #2 = lower x variant, #3 = universal binder var
  \labsep{xa}{#1}{170}{2pt}
  \labsep{xb}{#2}{-170}{2pt}
  \labsep{ya}{\likex #3}{\coverylabelangle}{2pt}
  \labsep{ppxb}{\likex{\mkern-1mu\pp\mkern-.6mu #2}}{\coverppxlabelangle}{\coverppxlabeldist}
  \labsep{pya}{\likex{p #3}}{0}{2pt}
}
\newcommand\adddrinkercoverlabels[2]{\addalldrinkercoverlabels{#1}{#2}{y}}

\newcommand\drinkercoververtices{
  \begin{scope}[shift={(0,\drinkercovergap)}]
    \vx{\drinkerx,0}{xa}
    \vx{\drinkerxxx,-\drinkercoverseconddown}{ya}
    \vx{\drinkerxxxx,0}{pya}
  \end{scope}
  \vx{\drinkerx,0}{xb}
  \vx{\drinkerxx,-\drinkercoverfirstdown}{ppxb}
}

\newcommand\drinkercover{
  \drinkercoververtices
  \drinkercoveredges
}

\newcommand\drinkerfib{
\begin{scope}[shift={(0,\fibheight)}]
  \drinkercover
\end{scope}
\drinkerbase
\drinkerfibres
}

\newcommand\drinkerfiblabelledpair[2]{% #1 = upper x variant, #2 = lower x variant
  \drinkerfib
  \adddrinkercoverlabels{#1}{#2}
}

\newcommand\drinkerfibvertices{
  \drinkerbasevertices
  \begin{scope}[shift={(0,\fibheight)}]\drinkercoververtices\end{scope}
}

\newcommand\drinkerfibverticesandfibres{
  \drinkerfibvertices
  \drinkerfibres
}

\newcommand\drinkerbindingfiblabelled[2]{% #1 = upper x variant, #2 = lower x variant
  \drinkerfibverticesandfibres
  \adddrinkercoverlabels{#1}{#2}
  \de{ya}{pya}
  \de{xb}{ppxb}
  \de{y}{py}
  \de{x}{ppx}
}


\newcommand\figdrinkerbifib{\begin{figure*}%
\begin{center}\begin{tikzpicture}[graph]
  \begin{scope}[shift={(-5,0)}]\drinkerfiblabelledpair{x}{x}\end{scope}
  \begin{scope}[shift={(0,0)}]\drinkerbindingfiblabelled{x}{x}\end{scope}
  \begin{scope}[shift={(5,0)}]\drinkerfib\end{scope}
\end{tikzpicture}\end{center}\caption{\label{fig:drinkerbifib}A skew bifibration (left), its binding fibration (centre), and its skeleton (right).}\figrule\end{figure*}}



%%%%%%%  FIGURE:  fonet + leap graph

% macros for dualizer assignment
\newcommand\shortmapsto{\scalebox{.8}[1]{\begin{math}\mapsto\end{math}}}
\newcommand\assign[2]{#1\mkern1.5mu\shortmapsto\mkern.7mu #2}
\newcommand\assignopen{\{\mkern.5mu}
\newcommand\assignclose{\mkern.5mu\}}
%\newcommand\twolinkassignment{\assignopen\assign x z,\assign y{fz}\assignclose}
\newcommand\twolinkassignment{\sublist{\subst x z,\subst y{fz}}}

\newcommand\twolinkfographvertices{
  \begin{scope}[shift={(-\halfedgelen,0)}]
    \lvx{-\edgelen,\halfedgelen}{x}{x}{180}
    \redlvx{0,\halfedgelen}{ppx}{\ppx}{90}
    \lvx{-\edgelen,-\halfedgelen}{y}{y}{180}
    \bluelvxsep{0,-\halfedgelen}{qqy}{\likex\qqy}{-90}{5pt}
  \end{scope}
  \begin{scope}[shift={(\halfedgelen,0)}]
    \redlvx{0,\halfedgelen}{pz}{\pz}{90}
    \bluelvxsep{0,-\halfedgelen}{qfz}{\likex{\qfz}}{-90}{5pt}
  \end{scope}
  \begin{scope}[shift={(\edgelen,0)}]
     \lvx{\halfedgelen,0}zz0
  \end{scope}
}
\newcommand\twolinkfographedges{
  \e x {ppx}
  \e y {qqy}
  \e {pz} {qfz}
}
\newcommand\twolinkfograph{\twolinkfographvertices\twolinkfographedges}

\newcommand\twolinkleapgraphedges{
  \begin{scope}[leap]
    \draw (ppx) to[out=-30,in=-150] (pz);
    \draw (qqy) to[out=30,in=150] (qfz);
    \draw (x) to[out=65,in=95,looseness=1.5] (z);
    \draw (y) to[out=-65,in=-95,looseness=1.5] (z);
  \end{scope}
}
\newcommand\twolinkleapgraph{
  \twolinkfographvertices
  \twolinkleapgraphedges
}

\newcommand\leapfig{%
\begin{figure}\begin{center}\vspace{-1ex}\begin{tikzpicture}[graph]
  \begin{scope}[xshift=-2.5cm]\twolinkfograph\end{scope}
  \begin{scope}[xshift=2.5cm]\twolinkleapgraph\end{scope}
\end{tikzpicture}\vspace{-4ex}\end{center}%
\caption{\label{fig:leap}A fonet (left) with
    dualizer $\protect\twolinkassignment$
    and its
    leap graph (right).}\end{figure}}



%%%%%%%%%%   PICTURE:  illustrating why we must collapse twins during contraction


\newcommand\rawunaryruleright[3]{\prftree[r]{\,$#1$}{#2}{#3}}


\newcommand\exppxfib[3]{% #1 vertex colour eg red blue, #2 cover binder name, #3 cover literal name
  \symb{(}
  \fibre{ex} {\exists x}{
    \over{black vx}{\hi}{#2}
  }
  \symb{\mkern2mu}
  \fibre{ppx} {\ppx}{
    \over{#1 vx}{\lo}{#3}
  }
  \symb{)}
}
\newcommand\aypyfib[1]{% #1 vertex colour eg red blue
  \fibre{a} {\forall y}{
    \over{black vx}{\lo}{al}
  }
  \symb{\mkern2mu}
  \fibre{py} {\py}{
    \over{#1 vx}{\lo}{pyl}
  }
}
\newcommand\expxexpxfib{
  \exppxfib{blue}{bl}{ppxl}
  \symb{\mkern-2mu\wedge\mkern-2mu}
  \exppxfib{red}{br}{ppxr}
  \draw (bl) -- (br);
  \draw (ppxl) -- (br);
  \draw (bl) -- (ppxr);
  \draw (ppxl) -- (ppxr);
  \draw (bl) -- (ppxl);
  \draw (br) -- (ppxr);
}


\newcommand\opCegHyp{
  \begin{tikzpicture}
  \fibration{
    \expxexpxfib
    \opgap
    \aypyfib{red}
    \opgap
    \aypyfib{blue}
  }  
  \end{tikzpicture}
}
\newcommand\opCegConc[1]{% #1 optional extra hi vertex, making illegal twin over \forall y
  \rule{0ex}{6.5ex}
  \begin{tikzpicture}
  \fibration{
    \expxexpxfib
    \opgap
    \fibre{a} {\forall y}{
      \over{black vx}{\lo}{al}
      #1
    }
    \symb{\mkern2mu}
    \fibre{py} {\py}{
      \over{blue vx}{\lo}{pyl}
      \over{red vx}{\hi}{pyh}
    }
  }
  \end{tikzpicture}
}
\newcommand\opCegConcGood{\opCegConc{}}
\newcommand\opCegConcBad{\opCegConc{\over{black vx}{\hi}{ah}}}
\newcommand\opgap{\symb{\hspace{3ex}}}


\newcommand\twincollapsepic{%
\begin{center}\begin{math}\begin{array}{@{}cc@{}}
%
\vctr{\rawunaryruleright{\textsf{C}}{\opCegHyp}{\opCegConcBad}}
&
\vctr{\color{red!80!black}\cross}
\\[11ex]
\vctr{\rawunaryruleright{\textsf{C}}{\opCegHyp}{\opCegConcGood}}
&
\vctr{\color{green!50!black}\checkmark}
\end{array}\end{math}\end{center}}




%%%%%%  PICTURE:  Peirce's law cp

\newcommand\peirceformula{\left(\rule{0ex}{1.7ex}\mkern-1mu(\pp\mkern-1mu\tightvee\mkern-3mu q)\mkern-2mu\tightwedge\mkern-2mu\pp\right)\mkern-1mu\tightvee\mkern-0mu p}
\newcommand\peirceimpliesformula{\left(\rule{0ex}{1.7ex}\mkern-1mu(p\mkern-1mu\tightimplies\mkern-2mu q)\mkern-1mu\tightimplies p\right)\mkern-2mu\tightimplies\mkern-.3mu p}

\newcommand\peircexone{-.9}
\newcommand\peircextwo{-.3}
\newcommand\peircexthree{.35}
\newcommand\peircexfour{1.05}
\newcommand\peircefibheight{1.8}
\newcommand\peircecoverradius{.22}%{.3}
\newcommand\peircey{-.14}
\newcommand\peircecp{
  \begin{scope}[shift={(0,\peircefibheight)}]
    \begin{scope}
      % cover vertices (coloured)
      \begin{scope}
        \redvx{\peircexone,0}{cppl}
        \bluevx{\peircexthree,0}{cppr}
      \end{scope}
      \begin{scope}[shift={(0,-\peircecoverradius)}]
        \bluevx{\peircexfour,0}{cpd}
      \end{scope}
      \begin{scope}[shift={(0,\peircecoverradius)}]
        \redvx{\peircexfour,0}{cpu}
      \end{scope}
    \end{scope}
  \end{scope}
  % base vertices
  \lvxd{\peircexone,0} {ppl} {\pp}
  \lvxd{\peircextwo,\peircey}{q}{q}
  \lvxd{\peircexthree,0}{ppr}{\pp}
  \lvxd{\peircexfour,0}{p}{p}
  % cover edges
  \e {cppl} {cppr}
  % base edges
  \e {ppl} {ppr}
  \e {q} {ppr}
  % fibration edges
  \begin{scope}[fibres]
    \fe {cppl}{ppl}
    \fe {cppr}{ppr}
    \fe {cpu}{cpd}
    \fe {cpd}{p}
  \end{scope}
}
