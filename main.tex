\documentclass{article}
\usepackage[utf8]{inputenc}
\usepackage{ebproof}
\usepackage{amsmath}
\usepackage{ebproof}
\usepackage{virginialake}

\parindent=0pt

\usepackage{amsthm}\theoremstyle{plain}
\newtheorem{thm}{Theorem}[section]

\theoremstyle{definition}
\newtheorem{definition}[thm]{Definition}
\newtheorem{lemma}[thm]{Lemma}
\newtheorem{notation}[thm]{Notation}
\newtheorem{proposition}[thm]{Proposition}
\newtheorem{theorem_}[thm]{Theorem}
\newtheorem{example}[thm]{Example}
\newtheorem{remark}[thm]{Remark}

\title{On first-order combinatorial proofs}
\author{Jui-Hsuan Wu, Lutz Strassburger}

\begin{document}

\maketitle

\section{From skew bifibration to contraction/weakening}

\begin{theorem_}
Let $A$ and $B$ be two formulas and $f: G(A) \rightarrow G(B)$ a skew bifibration. Then there exists a derivation $\od{\odd{\odh {A}}{\Delta}{B}{\{\wD, \cD\}}}$.
\end{theorem_}

$f$ can be seen as a skew fibration from $G(A^P)$ to $G(B^P)$, which gives the existence of the propositions $A'$ and $B'$, and of the following derivation:
  \[\od{\odd{\odd{\odd{\odh{A^P} }
  {\Delta }{A'}{\me} }
  {\Delta' }{B'}{\acD} }
  {\Delta''}{B^P}{\wD}} \]

\begin{lemma} there exists $B''$ such that $B''^P$ = $B'$.

\begin{proof}
Consider the dervation $\Delta''$. If some $U_x$ (or $E_x$) is introduced
via weakening, then all the atoms it binds in $B^P$ should also be introduced 
via weakening. In fact, an atom of $B^P$ is introduced via weakening is 
equivalent to the fact that its corresponding vertex is not in the image of $f$. 
Since there is an edge from $U_x$ (resp. $E_x$) to all the literals it binds in the 
binding graph $\overrightarrow{G(B)}$, if one of the atoms is in the image, 
$U_x$ (resp. $E_x$) should also be in the image since $f$ is a fibration on binding graphs.

This means that a such $B''$ can be obtained from $B$ by erasing all the $U_x$ and $E_x$ introduced via weakening and all the atoms they bind.
\end{proof}
\end{lemma}

We introduce new (atomic) symbols $E_x^*$ and $U_x^*$ which are used to
represent disjunctions of $E_x$ and $U_x$ respectiveley.

We define a translation $(\cdot)^*$ inductively by:
\begin{itemize}
  \item $(E_x \vee \cdots \vee E_x)* = E_x^*$
  \item $(U_x \vee \cdots \vee U_x)* = U_x^*$
  \item structural recursion in all the other cases.
\end{itemize}

Then the derivation:
 \[\od{\odd{\odd{\odh{A^P} }
  {\Delta }{A'}{\me} }
  {\Delta' }{B'}{\acD}} \]
can be translated to the derivation:
\[\od{\odd{\odh{(A^P)^*} }
{\Delta^* }{(B')^*}{}} \]

where $\Delta^*$ is the derivation obtained by replacing all the formulas $F$
with $F^*$.

Note that we should also introduce new rules in the derivation $\Delta^*$.
Otherwise, the rule applications cannot work in general.

To do that, we define the following rule transformation:

\begin{center}
\mbox{
\begin{prooftree}
  \hypo{Q_x^*}
  \infer1[$\acD$]{Q_x^*}
\end{prooftree}

$\leadsto$

\begin{prooftree}
  \hypo{Q_x^*}
  \infer1[$=$]{Q_x^*}
\end{prooftree}

}
\vspace{0.4cm}

\mbox{
\begin{prooftree}
  \hypo{(Q_x^* \wedge C) \vee (Q_x^* \wedge D)}
  \infer1[$\me$]{Q_x^* \wedge (C \vee D)}
\end{prooftree}

$\leadsto$

\begin{prooftree}
  \hypo{(Q_x^* \wedge C) \vee (Q_x^* \wedge D)}
  \infer1[$\me'$]{Q_x^* \wedge (C \vee D)}
\end{prooftree}

}
	
\end{center}
where $Q_x^*$ stands for $E_x^*$ or $U_x^*$.

$\Delta^*$ can now be transformed into a valid derivation $\Delta_1$ by using the two
transformation rules above and by applying them in a bottom-up theoremstyle:
\[\od{\odd{\odh{(A^P)^*} }
{\Delta_1 }{(B')^*}{\acD, \me, \me'}} \]

\begin{lemma}
Every line of $\Delta_1$ is a propositional encoding (by
confounding $Q_x^*$ with $Q_x$).

\begin{proof}
We proceed by bottom-up induction in the derivation.
Clearly, $B'*$ is a propositional encoding as there is no disjunction of $Q_x$
in it.

Now consider the $\me$ rule:
\begin{prooftree}
  \hypo{(C \wedge D) \vee (E \wedge F)}
  \infer1[$\me$]{(C \vee E) \wedge (D \vee F)}
\end{prooftree}

Suppose that $(C \vee E) \wedge (D \vee F) = G^P$ for some $G$.
Since $C \vee E$ cannot be $Q_x$ (otherwise, the rule applied would be
$\me'$), $G$ can be written as $G_1 \wedge G_2$ with $C \vee E = G_1^P$ and $D
\vee F = G_2^P$.

We have thus $G_i = \forall x_i H_i$ or $J_i \vee K_i (i = 1, 2)$.

If $G_i = \forall x H_i$ for some $i$, then there will a conjunction of $U_x*$
and some formula which can never be eliminated by the rules $\me$, $\me'$ and
$\acD$. However, there exists no such conjunction in ${A^P}^*$, which leads to a
contradiction.

Hence, $G_i$ can be written as $J_i \vee K_i$ for $i = 1, 2$. We now have $(C
\wedge D) \vee (E \wedge F) = ((J_1 \wedge J_2) \vee (K_1 \wedge K_2))^P$.

\end{proof}	
\end{lemma}
\end{document}
