\documentclass{article}
\usepackage[utf8]{inputenc}
\usepackage{ebproof}
\usepackage{amsmath}
\usepackage{ebproof}
\usepackage{virginialake}

\parindent=0pt

\usepackage{amsthm}
\theoremstyle{plain}
\newtheorem{thm}{Theorem}[section]

\theoremstyle{definition}
\newtheorem{definition}[thm]{Definition}
\newtheorem{lemma}[thm]{Lemma}
\newtheorem{notation}[thm]{Notation}
\newtheorem{proposition}[thm]{Proposition}
\newtheorem{theorem_}[thm]{Theorem}
\newtheorem{example}[thm]{Example}
\newtheorem{remark}[thm]{Remark}

\title{On first-order combinatorial proofs}
\author{Jui-Hsuan Wu, Lutz Strassburger}

\begin{document}

\maketitle

\section{From skew bifibration to contraction/weakening}

\begin{theorem_}
Let $A$ and $B$ be two formulas and $f: G(A) \rightarrow G(B)$ a skew bifibration. Then there exists a derivation $\od{\odd{\odh {A}}{\Delta}{B}{\{\wD, \cD\}}}$.
\end{theorem_}

\begin{proof}
$f$ can be seen as a skew fibration from $G(A^P)$ to $G(B^P)$, which gives the existence of the propositions $A'$ and $B'$, and of the following derivation:
  \[\od{\odd{\odd{\odd{\odh{A^P} }
  {\Delta }{A'}{\me} }
  {\Delta’ }{B'}{\acD} }
  {\Delta’’}{B^P}{\wD}} \]

Now we prove that there exists $B''$ such that $B''^P$ = $B'$.

We consider now the dervation $\Delta''$. If some $U_x$ (or $E_x$) is introduced
via weakening, then all the atoms it binds in $B^P$ should also be introduced 
via weakening. In fact, an atom of $B^P$ is introduced via weakening is 
equivalent to the fact that its corresponding vertex is not in the image of $f$. 
Since there is an edge from $U_x$ (resp. $E_x$) to all the literals it binds in the 
binding graph $\overrightarrow{G(B)}$, if one of the atoms is in the image, 
$U_x$ (resp. $E_x$) should also be in the image since $f$ is a fibration on binding graphs.

This means that a such $B''$ can be obtained from $B$ by erasing all the $U_x$ and $E_x$ introduced via weakening and all the atoms they bind.

\end{proof}
\end{document}
