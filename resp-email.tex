----------------------- REVIEW 1 ---------------------
SUBMISSION: 51
TITLE: Combinatorial Proofs and Decomposition Theorems for First-order Logic
AUTHORS: Dominic Hughes, Lutz Straßburger and Jui-Hsuan Wu

----------- Overall evaluation -----------
This paper presents a new proof system, called combinatorial proofs, for FO (without the equality predicate).
The author(s) shows that it is complete and sound and "syntax free".
Intuitively, combinatorial proofs are proofs/derivations that are represented as graph homomorophisms.
This is in contrast with traditional systems (such as sequent calculus, resolution, etc)
which are syntax based.

The motivation is that combinatorial proofs can be used as the notion for equivalence between derivations in different systems.
That is, two derivations in different systems can be considered ``equal'' if they are translated into the same combinatorial proof.

Similar result has been established by Hughes in [18], which is published as ArXiv paper.
But the proof in the current paper is much simpler.
It also allows a translation from combinatorial proofs back to syntactic proofs.
It should be stressed that the paper only considers FO without the equality predicate.

The outline of the paper is as follows.
Sect. II introduces sequent calculus which is known to be complete and sound.
Sect. III introduces fographs, which is a way of representing FO formulas as graphs.
Sect. IV introduces fonets and combinatorial proofs, where fonets themselves are graph representations of derivations.
Sect. V introduces a new system called KS1.
Sect. VI contains the statement of the main results (Theorem 19) along with its decomposition result for the system KS1.
Roughly, the decomposition result states that every derivation in KS1 can be decomposed into two parts:
the first part uses only the "linear fragment" of KS1 and the second part uses the rules outside the linear fragment and the equivalent rule \equiv.
Theorem 19 states that combinatorial proofs are complete and sound,
where completeness is obtained by applying the decomposition result and show that each part of the derivations can be translated into Combinatorial proofs.
Sect. VII presents the translation between the sequent calculus and KS1.
Sect. VIII shows the construction between fonets and proofs using only the linear fragment of KS1.
Sect. IX shows the construction between fonets and proofs using rules outside the linear fragment.

I find the result interesting and I think it fits LICS well.
The technical presentation is reasonable enough for me to be convinced of its correctness.

I have some complaints though.

a) The presentation is very dry.
It is mainly a sequence of definitions, lemmas, theorems, proofs with very little justification/explanation in between.
For example, fonets, which I think is the most important part, are presented as a definition without any intuitive explanation
on how they can be used to represent proofs. (See below for more details.)
Some examples are presented in Figure 4 and 5, but I find them useless due to lack of explanation.

I also suggest the author(s) to provide some broad overview of the proof.
There is also no description of "the organization of the paper".
The readers are left to figure out themselves how sections in the paper connect to each other.

b) Some of the definitions, esp. in Sect. III and IV, are very confusing.
(See below for more details.)
I have to go back to [18] which use almost the same terminology and definition as the current paper,
and I find it easier to understand them in [18].

More detailed comments:

I think it should be stressed early (as early as in the abstract) that the paper is about FO without the equality predicate,
since it is usually considered in standard proof systems.

p.3, left: The notion of colored graphs here does not require adjacent vertices to have different colors, but the one in [18] does.
Is there a mistake here, since the one in [18] requires different colors?

-> There is no mistake. The requirement mentioned here is never used in the
paper.

p.3, right: I suppose for fographs it is possible that different vertices have the same label?
I also find the definition of fographs a bit confusing, since it starts with abstract graph theoretic properties of fographs,
before presenting the construction of fographs from formulas.
I suggest to reverse it, by first presenting the construction of fographs (from formulas) before enumerating all the properties of fographs.
In fact, this is how forgraphs are presented in [18].

->

p3,right: What about the atoms True and False in the fographs?
They are part of the definition of FO formulas, but are they really needed?
Almost all results in the paper require that False does not appear in the proof.
Maybe they can be omitted.

-> The atoms are needed for establishing the equivalence between KS1 and MLS1*.
(TO CHECK)

p.4, left: The notion of "dual" is not defined yet.
Again, it is already defined nicely in [18].
I also suggest to present here some intuitive explanation on how fonets represent proofs.
The same for the other notions here such as links, leaps, (bi)fibration, skew (bi)fibration, and their relations with derivations in KS1 and LK1.
Figures 4 and 5 can be easily expand to provide intuitive explanation.

-> Yes, the notion of duality is implicitly suggested by the notation
\overline{p}.

p.6, Theorem 19: I think it is safe to assume that formulas are rectified, thus, there is no need for \hat{A} here.
Otherwise, there will be too many notations to remember here.

-> 

p.7, Lemma 22: \overline{S{A}} is not defined yet.
I assume it is the complement of the corresponding formula for S{A}?

-> Yes, it is.

p.10, right: A\cdot here is just "variable renaming"?

-> We use this notation to simplify the rectified version of rules for
quantifiers. Variable renaming is a more general notion as it allows rename more
than one variable.

p.10, left: Why "resource management" in the title of Sect. IX?

-> Like combinatorial proofs for propositional logic, the skew bifibration part
corresponds to the linear part of the sequent proof. That's why we mention
resource management in the title though it is not explicitly discussed in the
paper.

p.11, right: on full fograph homomorphism.
There is something missing here since the homomorphism is not necessarily injective?
Or, it is meant to be "strong" homomorphism?

-> The notion of fullness is well defined in category theory and that is exactly
what we need here.

----------------------- REVIEW 2 ---------------------
SUBMISSION: 51
TITLE: Combinatorial Proofs and Decomposition Theorems for First-order Logic
AUTHORS: Dominic Hughes, Lutz Straßburger and Jui-Hsuan Wu

----------- Overall evaluation -----------
This paper addresses an important question in proof theory -- when are two proofs considered equivalent. In this case, the logic in question is classical first-order logic. This paper builds on a notion of classical proofs, developed by Hughes, that is based on a graph structure called first-order nets (fonets). Greatly simplified, a fonet can be seen as 'a formula with links between dual atoms', with the formula part represented as a graph encoding series-parallel orders.
Hughes's technical report (cited as [19]) on this subject lays out much of the foundations for fonets, including a soundness and a completeness theorem w.r.t. a version of LK. However, there is no direct 'sequentialization' theorem for fonets, in the sense that it is not possible to directly extract a sequent proof from a fonet. This paper proposes a different approach to sequentialization by relating fonets, not with sequent calculus, but with a deep-inference proof system called KS1, formulated in the calculus of structures (CoS).

A key technical idea to obtain a tight correspondence between fonets and KS1 is to first obtain a decomposition result for KS1, which decomposes KS1 proofs into a 'linear' part, followed by a classical part in contraction and weakening are applied, and quantifier scopes are rearranged.
The correspondence between fonets and the linear part of the decomposed proof is then proved by adapting known techniques for proving the correspondence between first-order MLL (multiplicative linear logic) and fonets; whereas the classical part requires new proof techniques but is more manageable.

The sequentialization theorem is one of the important questions surrounding proofnet-like proof representations, so in this sense the result of the paper is of some significance. It is possible to have a proofnet calculi that have no corresponding sequent proofs (e.g., Retore's pomset logic). In this case, it is unlikely sequentialization, in the sense of directly translating fonet to cut-free sequent proofs, is possible. A compromise here is to use a different proof calculus, based on deep inference. In principle, we can translate deep inference proofs to cut-free sequent proofs, using cuts and cut-elimination, but this would likely obscure the structures of the original fonets.

Another interesting aspect of the paper is that it contains probably one of the more interesting applications of the decomposition theorem for calculus of structures. The technical results are believable. I didn't check all the proofs, but a lot of intermediate lemmas, particularly those dealing with decomposition are mostly proved along the standard lines in this area so I don't particularly doubt their correctness.

On the negative side: there is still no theory of cut-elimination for fonets.
The paper is very dense, with lots of technical lemmas that are hand-waived through. I think it might be better to just omit some of these lemmas and dedicate more space to provide intuitive examples. A slight nitpick on notations for proofs/derivations: the vertical figures for derivations of CoS waste a lot of space that could have been used to improve presentation.

Overall this looks like a solid paper with an interesting and non-trivial result. I recommend acceptance.

Minor comments for the author(s):
- Lemma 30: you cite [35] for the proof of this lemma, but as far as I know [35] does not discuss first-order quantifiers. I can see a possibility of quantifiers interferring with the atomic interaction rule. For example,

(forall x. A) \/ B
-------------------
forall x.(A \/ B)
---------------------------------
forall x.(A \/ (px \/ ~px) /\ B))

where x is vacuous in B. It doesn't seem straightforward that ai can be permuted directly above structural equivalence.

-> 

----------------------- REVIEW 3 ---------------------
SUBMISSION: 51
TITLE: Combinatorial Proofs and Decomposition Theorems for First-order Logic
AUTHORS: Dominic Hughes, Lutz Straßburger and Jui-Hsuan Wu

----------- Overall evaluation -----------
The paper is concerned with canonically representing first-order logic proofs in various proof systems in a more abstract system of "combinatorial proofs". This is a fundamental question in proof theory and hence in computer science. The contribution of the paper is a new deep inference system KS1  for first-order logic. KS1 is closer to combinatorial proofs and the paper provides simpler and more complete results linking the two systems.

  The paper is well written and an enjoyable read. Though not an expert, I believe the contribution is significant and deserves acceptance at LICS.
